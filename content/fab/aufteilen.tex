\section{Dokument Aufteilen \& Quellcodedarstellung}
\begin{frame}{Ein Dokument aufteilen}
\lstsettex
	\begin{Code}
	\centering
	\begin{minipage}{0.7\linewidth}
	\lstinputlisting[linerange=31-37]{./listings/commands.tex}
	\end{minipage}
	
	\end{Code}
\end{frame}
\begin{frame}{Quellcode}
	 benötigt wird das \textbf{listings} oder das \textbf{listingsutf8} package
\lstsettex
	\begin{Code}
	\centering
	\begin{minipage}{0.9\linewidth}
	\lstinputlisting[linerange=38-53,caption={Darstellung festlegen}]{./listings/commands.tex}
	\end{minipage}	
	\end{Code}
\end{frame}
\begin{frame}{Beispiel}
\lstsettex
	\begin{Code}
	\centering
	\begin{minipage}{0.9\linewidth}
	\lstinputlisting[linerange=54-62,caption={Quellcode einbinden}]{./listings/commands.tex}
	\end{minipage}	
	\end{Code}
\end{frame}

\begin{frame}{So sieht's im pdf aus}
\lstsetjava
\begin{Code}
	\centering
	\begin{minipage}{0.9\linewidth}
	\lstinputlisting[linerange=63-68,caption={This is Java}]{./listings/commands.tex}
	\end{minipage}	
	\end{Code}
\end{frame}