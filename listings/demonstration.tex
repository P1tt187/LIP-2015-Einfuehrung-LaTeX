% ich bin ein kommentar 
\documentclass{scrartcl} %brief, artikel, report, buch, präsentation
%hier ist die preambel, hier wird die allgemeine Formatierung festgelegt
\usepackage[utf8]{inputenc} % zeichenkodierung (wichtig für umlaute)
\usepackage[T1]{fontenc}
\usepackage[ngerman]{babel} %deutsche bezeichner für verzeichnisse
\usepackage{amsmath} %formeln 
\title{Ein Testdokument}
\author{Max Mustermann}
\date{\today}
\begin{document} %hier geht das dokument los 
\maketitle %titelseite
\tableofcontents %inhaltsverzeichnis
\section{Einleitung}
 
Hier kann ein beliebig langer text stehen.       Überflüssige Leerzeichen werden ignoriert
,toll , oder?
 
Ich bin ein neuer absatz.

\subsection{Ein Unterpunkt}
 \LaTeX{} nummeriert automatisch und fügt neue punkte dem Inhaltsverzeichnis hinzu.
Formeln kann \LaTeX ebenfalls darstellen. Hierzu gibt es einen Mathematik Modus in dem eigene befehle gelten
Zwei von Einsteins berühmtesten Formeln lauten:
\begin{align} %hier fängt der mathe modus an
E &= mc^2                                  \\
m &= \frac{m_0}{\sqrt{1-\frac{v^2}{c^2}}}
\end{align}
Aber wer keine Formeln schreibt, braucht sich \\ %erzwungener zeilenumbruch
damit auch nicht zu beschäftigen.
Mathe geht auch inline, sieht allerdings etwas anders aus
 $ m = \frac{m_0}{\sqrt{1-\frac{v^2}{c^2}}} $
\end{document}
